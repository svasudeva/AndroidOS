\documentclass{beamer}

\mode<presentation>
{
  \usetheme{Dresden}
  \setbeamercovered{transparent}
}

\usepackage[english]{babel}

\usepackage{listings}
\usepackage{hyperref}

\title [ Android Operating System] % short title
{Mobile Operating Systems}

\subtitle
{ANDROID}

\author[Sheree Vasudeva] % (short authors list)
{Sheree Vasudeva}


\begin{document}

\begin{frame}
  \titlepage
  \begin{center}
    \includegraphics[height=3.0cm]{diagrams/ics_0}
  \end{center}
\end{frame}


% SLIDE


\begin{frame}{Outline}
  \begin{itemize}
    \item Introduction
	  \begin{itemize}
   		\item What is Android?
    		\item History of Android
    		\item Android Versions
    		\item Features of Android
	  \end{itemize}	
    \item Android Architecture
	  \begin{itemize}
		\item Architecture Details
	  \end{itemize}	
    \item MultiTasking
	  \begin{itemize}
    	  \item MultiTasking-IPHONE/IPAD
    	  \item MultiTasking-The Android Way
	  \item Processes and Threads
	  \end{itemize}
    \item Others
  \end{itemize}	

\end{frame}


% SLIDE
\begin{frame}{What is Android?}
  \begin{itemize}
    \item Word "Android" means a robot,designed to look and act like a human
    \item  Android is a software stack for mobile devices that includes an operating system, middleware and key applications. 
    \item It is based on 2.6x LINUX kernel.
    \item It is developed by the Open Handset Alliance,a consortium of software, hardware and telecommunication companies led by Google
    \item The Android SDK provides the tools and APIs necessary to begin developing applications on the Android platform using the Java programming language.
  \end{itemize}
\end{frame}

% SLIDE
\begin{frame}{History of Android}
  \begin{itemize}
    \item Android Inc was founded in October 2003 by Andy Rubin, Rich Miner, Nick Sears and Chris White
    \item Later in August 2005, Google acquired Android Inc employing its founders
    \item OHA led to open standards for mobile software and hence most of Android's source code is published under Apache License
    \item T-Mobile G1 is the first ever Android mobile device which was launched in 2008
    \item Currently there are 200 million android activations
  \end{itemize}
\end{frame}

% SLIDE
\begin{frame}{Android Versions}
  \begin{itemize}		
    \item 4.0 Ice Cream Sandwich	
    \item 3.x.x Honeycomb		
    \item 2.3.x Gingerbread		
    \item 2.2 Froyo			
    \item 2.1 Eclair			
    \item 1.6 Donut			
    \item 1.5 Cupcake			
    \item Two more internal release named "Astro" and "Bender",which were changed from Robots to Desserts to avoid trademark issues.
  \end{itemize}
\end{frame}

% SLIDE
\begin{frame}{Features of Android}
  \begin{itemize}
    \item Web Browser:Integrated browser based on the open source WebKit engine
    \item User Interface: Custom 2D graphics library; 3D graphics based on the OpenGL ES 2.0 specifications
    \item Data Storage: SQLite for structured data storage
    \item Media support: Common audio, video, and still image formats (MPEG4, H.264, MP3, AAC, AMR, JPG, PNG, GIF)
    \item Connectivity: Bluetooth, EDGE, 3G, and WiFi
    \item Hardware Support: Camera, GPS, compass, and accelerometer, barometer
    \item ScreenCapture: Latest Version lets you take a screenshot of the screen by pressing the power and volume buttons together.
    \item Rich development environment including a device emulator and a plugin for the Eclipse IDE
  \end{itemize}
\end{frame}


% SLIDE
\begin{frame}{Android Architecture}
  \begin{center}
    \includegraphics[height=5.0cm]{diagrams/cpubound}
  \end{center}
\end{frame}

% SLIDE
\begin{frame}{Architecture Details}

  \begin{itemize}
  \item Applications
  	\begin{itemize}
  	\item Android contains a set of core applications including an email client, SMS program, calendar, maps, browser and contacts. All are written using Java.
	\end{itemize}
  \end{itemize}

  \begin{itemize}
  \item Application Framework
  	\begin{itemize}
  	\item Rich and extensible views to build applications including grids, burtons, textbooks,etc.
  	\item Content providers provide access to data from other devices like contacts.
  	\item Resource manager provides access to non code data like graphics and layouts
  	\item Activity manager manages the navigations and lifecycle of applications
  	\item Notification manager enables custom alerts in the status bar.
	\end{itemize}
  \end{itemize}

\end{frame}

% SLIDE
\begin{frame}{�Continued}
  \begin{itemize}
  \item Application Libraries
  	\begin{itemize}
		\item System C library - a BSD-derived implementation of the standard C system library (libc), tuned for embedded Linux-based devices
		\item Media Libraries -  the libraries support playback and recording of many popular audio and video formats, as well as static image files.
		\item Surface Manager - manages access to the display subsystem and seamlessly composites 2D and 3D graphic layers from multiple applications
		\item LibWebCore - a modern web browser engine which powers both the Android browser and an embeddable web view SGL - the underlying 2D graphics engine
		\item 3D libraries - an implementation based on OpenGL ES 1.0 APIs; the libraries use either hardware 3D acceleration (where available)
		\item FreeType - bitmap and vector font rendering
		\item SQLite - a powerful and lightweight relational database engine available to all applications
	\end{itemize}
  \end{itemize}

\end{frame}

% SLIDE
\begin{frame}{Android Runtime: Dalvik Virtual Machine}
  \begin{itemize}
	\item DVM(Dalvik Virtual Machine) is the software for the mobile devices.
	\item Each application in the device runs as a separate user with its unique LINUX user ID.
	\item It has its own instance of DVM. Thus, device can run multiple VMs efficiently.
	\item JAVA source is then compiled to bytecode. Then it is converted to .dex(executable files) from JAVA .class files.
	\item Java bytecode is also converted into an alternative instruction set used by the Dalvik VM.
	\item Advantage is to make it compatible with systems having less memory and processor speed

  \end{itemize}

\end{frame}

% SLIDE
\begin{frame}{...Dalvik Virtual Machine}
  \begin{itemize}
	\item The Dalvik VM relies on the Linux kernel for underlying functionality such as threading and low-level memory management.
	\item Unlike Java VMs, which are stack machines, the Dalvik VM is a register-based architecture.
	\item As of Android 2.2, Dalvik has a just-in-time compiler.
	\item Compressed .dex files take less space than compressed JAVA .jar files
	\item Standard Java bytecode executes 8-bit stack instructions. Local variables must be copied to or from the operand stack by separate instructions.
	\item Dalvik instead uses its own 16-bit instruction set that works directly on local variables. 
	\item The local variable is commonly picked by a 4-bit 'virtual register' field. This raises its interpreter speed.
  \end{itemize}

\end{frame}

% SLIDE
\begin{frame}{Android Kernel}
  \begin{itemize}
    	\item It is a fork of LINUX kernel
	\item All the basic OS operations like I/O, memory management, and so on, are handled by the Linux kernel
	\item Binder is an IPC mechanism, one Android process can call a routine in another Android process, using binder to identify the method to invoke and pass the arguments between processes.
	\item Uses OOM for killing processes when memory is low
	\item Wakelocks are used for power management.Earlier it was a set of patches to the Linux kernel to allow a caller to prevent the system from going to low power state.
  \end{itemize}
\end{frame}


% SLIDE
\begin{frame}{MultiTasking-IPHONE/IPAD}
  \begin{itemize}
    	\item Mobile devices and tablets have limited memory for use. So,it is very important to manage this memory well.
	\item No hard disk means no swap memory. No swap and fixed memory
	\item Hence, the iphone and ipad systems enforce a policy whereby once an application leaves the foreground, it terminates i.e. no background processes can run.
	\item The iPad and iPhone user interfaces are single window, single document
	\item Yes,music players do multitask but they are not third party
	\item What in reality is not supported is the ability for third-party applications to multitask.
  \end{itemize}
\end{frame}

% SLIDE
\begin{frame}{MultiTasking-The Android Way}
  \begin{itemize}
    	\item Unlike either JAVA or .NET frameworks, the Android run time also manages the process lifetimes
	\item Android has a process virtual machine(DVM)
	\item Each process has its own virtual machine (VM), so an application's code runs in isolation from other applications
	\item Android ensures application responsiveness by stopping and killing processes as necessary to free resources for higher-priority applications
	\item Bundles which are applications enable to save the current state. So,if the process is killed and requested again, the user will be shown the saved bundle
  \end{itemize}
\end{frame}

% SLIDE
\begin{frame}{Processes and Threads}

  \begin{itemize}
 	\item The question is how is the priority of which application to kill decided?
	\item Applications are killed using LRU(Least Recently Used) algorithm when memory is required by other processes
	\item If processes have the same priority, the process that has been at the lowest priority longest is killed
	\item Process priority is also affected by interprocess dependencies.
	\item If an application has a dependency on a Service or Content Provider supplied by another application, the secondary application will have at least as high a priority as the application it supports
  \end{itemize}
\end{frame}

% SLIDE
\begin{frame}{�Continued}

  \begin{center}
    \includegraphics[height=5.0cm]{diagrams/processimage}
  \end{center}

\end{frame}

% SLIDE
\begin{frame}{�Process States}


  \begin{itemize}
	\item Active (foreground) processes are those hosting applications with components currently interacting with the user.
	\item Background Processes Processes hosting Activities that aren�t visible and that don�t have any Services that have been started are considered background processes.
	\item Visible, but inactive processes are those hosting �visible� Activities. As the name suggests, visible Activities are visible, but they aren�t in the foreground or responding to user events.
	\item Processes hosting Services that have been started. Services support ongoing processing that should continue without a visible interface. 
	\item Empty Processes improve overall system performance, Android often retains applications in memory after they have reached the end of their lifetimes. 
  \end{itemize}

\end{frame}


% SLIDE
\begin{frame}{Interesting Facts}

  \begin{itemize}
	\item Default File System: YAFFS(Yet Another Flash File System). Also, uses VFAT
	\item Data Storage can be internal, external, in a private database and on the web.
	\item The Android SDK includes a virtual mobile device emulator that runs on your computer
	\item The emulator lets you prototype, develop, and test Android applications without using a physical device except for making calls
	\item Popular Android apps: Android Market, Pandora, Angry Birds, Facebook, Google Maps.
  \end{itemize}

\end{frame}

% SLIDE
\begin{frame}{Interesting Links}
  \begin{itemize}
    \item \url{http://en.wikipedia.org/wiki/Android_(operating_system)}.
    \item \url{http://developer.android.com/guide/index.html}.
    \item \url{http://www.androidcentral.com/android-versions}.
    \item \url{http://blog.rlove.org/2010/04/why-ipad-and-iphone-dontsupport.html}
    \item \url{http://elinux.org/Android_Kernel_Features}
    \item \url{http://elinux.org/Android_Power_Management}
    \item \url{http://www.android.com/}
  \end{itemize}
\end{frame}

% SLIDE
\begin{frame}{The End}
  \begin{center}
    \includegraphics[height=5.0cm]{diagrams/android-vpn}
  \end{center}
Thank you!
\end{frame}

\end{document}

